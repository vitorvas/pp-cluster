\documentclass[twoside,a4paper,12pt,english]{inac17}

%INAC2017 SETUP: SET PAGE SIZE AND SET FOR USING graphicx PACKAGE.
\usepackage{graphicx}
\usepackage{babel,varioref,epsfig} %,rotating}
\usepackage{amssymb}
\usepackage[font=bf,center]{caption}
\usepackage{subfigure}

\title{PROFESSIONAL CLUSTER MANAGEMENT BY A SMALL SCIENTIFIC TEAM: CHALLENGES, SOLUTIONS
AND PERPECTIVES}

%INAC2017 SETUP: SPECIFY AUTHOR NAMES, AFFILIATION, ADDRESS AND E-MAIL.

\author{
  \bf{Vitor V. A. Silva, Andr\'e A. C. dos Santos and Renan O. Cunha}\\ \\
  CDTN - Centro de Desenvolvimento da Tecnologia Nuclear\\
  Av. Ant\^onio Carlos 6627 - Campus UFMG\\
  31270-901 - Belo Horizonte, MG\\
  \{vitors, aacs, roc\}@cdtn.br}

\begin{document}

%INAC2017 SETUP: PRINT TITLE
\maketitle

%INAC2017 SETUP: SETUP HEADS FOR PAGES
\pagestyle{myheadings}
\thispagestyle{empty}
\markboth{}{}


%INAC2017 SETUP: SET FIRST PAGE WITH NO PAGE NUMBER
\thispagestyle{empty}

%--------------------------------------------------------------------------------------

\begin{abstract_full_paper}
  The specification, configuration and management of a professional computer cluster are specialized
tasks usually hold by well trained teams, often full-time hired computer scientists. However, in
many situations and for widely different reasons, these very specific technical tasks must
be carried on by no other than the user itself. This is the situation at Centro de Desenvolvimento
da Tecnologia Nuclear - and in many nuclear research and educational centres in developing countries -
where the scientists are the users of the cluster but also the technical
team responsible to keep the system running. This paper presents the process of planning
and installing the whole operational system and scientific software of a professional cluster
aimed to be used in the nuclear engineering field from the point of view of its users.
The drawbacks of lack of expertise and technical skills to
manage such type of technology are opposed to the advantages of freedom to chose the solutions
which best fit to the problems to be solved. The details of selected methods or technologies
chosen for addressing a specific matter are presented together with other possible options, 
offering a broader view of the whole process of cluster's configuration. Specificities
of dealing with closed, restricted and open software, common in the nuclear engineering field,
are also put in perspective. The ideas and solutions presented in this paper can be a
valuable reference to other research teams found in a similar situation:
being scientists and its own technical staff at the same time.
\end{abstract_full_paper}

%--------------------------------------------------------------------------------------

\section{INTRODUCTION}\label{int}

Nowdays computers are indispensable tools for scientists of any field.
Some areas of research heavily rely on computational power in order to solve
complex problems by use of demanding algorithms, methods and heuristics.
Current methods used for nuclear reactor calculations, both thermal-hydraulics
and neutronics can be more accurate or even only applicable by using
many cores or computers alltogheter.

A computer cluster consists of set of computers connected to work together. The difference of the
definition between a computer cluster and a computer grid is that usually grids are more
heterogeneus and used to a differente set of problems at the same time, while a computer
cluster is aimed to have each node solving the same problem.

To be able to have sets of computers working together and be able to control the utilization of the system,
it is absolutelly fundamental to have an operational system capable of offering interconnection, data sharing,
task schedule and user administration. The \textbf{de facto} operational system widely used in these situations
is Linux \cite{Linux}. After chosing a suitable operational system, its mandatory to evaluate the applications to
be used, the level of security to be enforced and, fundamentally, the dynamics of the operation of the system.
These factors will define the suitable solutions for different aspects of the system, ranging from data
sharing among nodes to options to update, backup and acess the system.


%--------------------------------------------------------------------------------------

\subsection{Context}

As aforementioned, the cluster presented in this paper is aimed to scientific computing.
Since the expression scientific computing has many meanings, depending on the context, a
clarification is necessary. This cluster will be utilized to run computational fluid dynamics
(CFD) simulations, neutronics simulations - both using deterministic and Monte Carlo \cite{MC}
methods and coupled calculations, involving the two disciplines. It is also natural to predict
its expansion in utilization to other fields depending on its demand, performance and availability.

The users of the system are mainly nuclear and mechanic engineers, scientists and graduate
and undergraduating students. These users have different levels of knowledge of the work
they must carry and also are unevenly skilled as system users. These differences must be taken
in account when planning and choosing the operational characteristics of the system.

From the perspective of the Centro de Desenvolvimento da Tecnologia Nuclear (CDTN), such system
is a fundamental asset. Computer power is, at the very end, power of calculation and any research
team with demand for intensive computing is a potential user and potential client of the cluster.
With this in mind, there is a real potential of increase of research of CDTN related to
scientific computing. An interesting perspective of the way of research can change with the
use of intensive scientific computing - in this case, the extreme computing - is elegantily
presented by Dongarra \cite{Dongarra2017}.

%Teste de todas as citações disponíveis no bibtex: Wu\cite{Wu2016}, Nagaya\cite{Nagaya2015}.

%------------------------------------------------------------------------------

\section{OBJECTIVE}

The objective of this paper is to present the challenges, problems and solutions proposed
to setup a professional computational cluster to carry heavy numerical calculations in a
context of a small research team.

%\begin{equation}
%  \nabla\cdot\mathbf{J} + \Sigma_{a} \Phi - s =0 .
%  \label{eq_1}
%\end{equation}

%------------------------------------------------------------------------------

\section{METHODOLOGY}


  A figura~\ref{fig:esquema-cluster} apresenta o sistema de forma simplificada.
  
\begin{figure}[h] % t forces top and b forces bottom: can be added to h, ex. [ht]
%  \centering\includegraphics[width=8.5cm,height=8.5cm]{images/esquema_cluster_edited_bw.png}
  \centering\includegraphics[scale=0.25]{images/esquema_cluster_edited_bw.png}
  \caption{Captions are flush with the left.}
  \label{fig:esquema-cluster}
\end{figure}

\begin{itemize}

\item (Indentificação do sistema $\rightarrow$ Características: memória, CPUs, GPUs (Graphics Processing Units), cores, rede, topologia, sistema de arquivos, inserção na rede institucional)

\item (Possíveis soluções/soluções encontradas em relação a cada característica apresentada. Por exemplo: Lustre \cite{Lustre} ou Gluster\cite{Gluster} para o sistema de arquivos?)

\item (Para cada solução escolhida: Por quê?)
  
\end{itemize}

The system is formed by a set of eight identical blade servers each with 20 cores and equiped with
NVIDIA GPU's M4000 CUDA\cite{CUDA} capable. The GPU's on servers, despite capable of visualization,
are expected to be used as secondary calculation units. Also part of the sistem, the master server is aimed to calculations
visualiation and simulations manager. The master computer has
four network devices and a NVIDIA GPU M2000 aimed to visualization and graphics display.

The original OS sold with the cluster is an OEM (Original Equipament Manufacturer) Windows 7 \cite{windows7}. This
OS is not supported by the whole set of software expected to be used by the research group users. The chosen
system to replace the original Windows 7 is CentOS \cite{centos}, which offers a plataform to natively run all
software needed by the users. However, there are more than one way of setup a cluster system with respect
to filesystem, software execution and user access.

The next paragraphs, describe different option of setup for a Linux cluster with respect to software execution.

\begin{itemize}
\item Direct OS installation
\item Use of dockers
\item SO virtualization
\end{itemize}

\subsection{Making use of graphics processors}

The system under consideration has powerful GPU cards installed in machines
which have no output display. The use of GPU's to perform tasks other than graphics processing
is nowdays widespread in many fields of research \cite{UsoDeGpus}. Modern software is written with
the objective of taking advantage of this ``extra'' processing capabilities and studies of software
engineering on parallelization of sequential algorithms to be applied in GPU's are a significant
nowdays.

In order to make the use of GPU's for general processing less dificult, libraries, compilers and
frameworks were developed. At the beggining these tools were developed and deployed by graphics cards
manufacturers. The pionner in offering both GPU's cards capable of general use is NVIDIA, which,
without surprise, is also pionner in making a available a full framework for its GPU programming.
This tool is called CUDA (Compute Unified Device Architecture) and is a parallel computing platform
and application programming interface (API) model.

However, with the increase of the use the cards, developers start to demand some
standardization ir order to be able experiment new algorithms in different cards. Today, the
reference framework for development for heterogeneus platforms is \textit{OpenCL} \cite{OpenCL}.

--- Como funciona o OpenCL

The idea for the cluster system, is to provide both CUDA libraries for
software developed for NVIDIA graphic cards and also OpenCL libraries to
be used by software written to make use of the programming genericity
provided by OpenCL. As the time of writing, the libraries were installed
in only one node of the cluster and tests are being envisaged to check
compatibility of both framewoks.

\section{RESULTS AND ANALYSIS}

\begin{itemize}

\item (Situação atual do sistema - no momento de finalizer o paper)
\item (Dificuldades identificadas: Resolvidas (como) e Pendentes.
  
\end{itemize}

%------------------------------------------------------------------------------
\subsubsection{Sub-subsection level and lower: only first character uppercase}

Figures and tables should appear as closely as possible to where they are first cited, e.g. Fig.~\ref{esquema-cluster}, in the text.  Figures are numbered in Arabic numerals, with the caption centered below the figure, in boldface. Double-space before the figure, and after the figure caption.


\newcommand{\cc}{\centering}
\newcommand{\rr}{\raggedright}
\newcommand{\tn}{\tabularnewline}
\renewcommand{\arraystretch}{1.5}
\begin{table}[h]
\caption{Numerical results to the model problem} %title of the table
\centering % centering table
\begin{tabular}{|p{3cm}|p{2cm}|p{2cm}|p{2cm}|p{2cm}|}
\hline
\cc Mesh             &\cc 8$\times$8  &\cc 16$\times$16   &\cc 32$\times$32   &\cc 64$\times$64 \tn \hline        
\rr Nodal            &\cc 1.000       &\cc 2.500          &\cc 6.250          &\cc 1.563        \tn \hline
\rr Characteristic   &\cc 1.000       &\cc 2.500          &\cc 6.250          &\cc 1.563        \tn \hline
\end{tabular}
\label{table_1}
\end{table}

When importing figures or any graphical image please verify two things:

\begin{itemize}

\item Any number, text or symbol is in Times font and is not smaller than 10-point after reduction to the actual window in your paper;

\item That it can be translated into PDF.

\end{itemize}



Tables, like Table~\ref{table_1}, are numbered in Arabic numerals, with the caption centered above the table, in {\bf boldface}.  Doublele-space before and after the table.

%------------------------------------------------------------------------------

\section{CONCLUSIONS}

(O que está bom, o que está ruim na atual instalação)
(Trabalhos futuros: pra onde ir (operação), possíveis mudanças)

Uma citação \cite{Henderson17}.

%------------------------------------------------------------------------------

\section*{Acknowledgments}
The authors would like to thank FUJB for financing the cluster acquisition
as part of the project XXXXX.

%%%%%%%%%%%%%%%%%%%%%%%%%%%%%%%%%%%%%%%%%%%%%%%%%%%%%%%%%%%%%%%%%%%%%%%%%%%%%%%%%%%%%%%%%%%%

%\begin{thebibliography}{99} %99 é o número máximo que o thebibliography permite. Numero de referencias que aparecerão.

%\bibitem{article} B. Author(s), ``Title", \textit{Journal Name in Italic}, \textbf{Volume in Bold}, pp. 34--89, (19xx).

%\bibitem{proceeding} C. D. Author(s), ``Article Title", \textit{Proceeding of Meeting in Italic}, Location, Dates of Meeting, Vol. n, pp. 134--156, (19xx).

%\bibitem{book} E. F. Author. \textit{Book Title in Italic}, Publisher, City \& Country (19xx).

%\bibitem{website} ``Spallation Neutron Source: The next-generation neutron-scattering facility for the United States", \verb#http://www.sns.gov/documentation/sns_brochure.pdf# (2002).

%\end{thebibliography}

% ---------------------------------------------------------
% Minha bibliografia usando arquivo externo

\bibliographystyle{unsrt}
\bibliography{bibli}


\end{document}
