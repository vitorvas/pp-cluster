\documentclass{anstrans}

%%%% packages and definitions (optional)
\usepackage{graphicx} % allows inclusion of graphics
\usepackage{booktabs} % nice rules (thick lines) for tables
\usepackage{microtype} % improves typography for PDF

% Para português
\usepackage[utf8]{inputenc}     % Codificacao do documento (conversão automática dos acentos)

\newcommand{\SN}{S$_N$}
\renewcommand{\vec}[1]{\bm{#1}} % vector is bold italic
\newcommand{\vd}{\bm{\cdot}} % slightly bold vector dot
\newcommand{\grad}{\vec{\nabla}} % gradient
\newcommand{\ud}{\mathop{}\!\mathrm{d}} % upright derivative symbol


%%%% changes from the original 'anstrans' class
\usepackage{fancyhdr} % allows headers and footers

\renewcommand\headrule{} % remove underline in the header
\setcounter{secnumdepth}{1}
\renewcommand{\thesection}{\Roman{section}.}
\renewcommand{\thesubsection}{\arabic{subsection}.}
\renewcommand{\thesubsubsection}{\Alph{subsubsection}.}
\makeatletter
\renewcommand*{\@seccntformat}[1]{\csname the#1\endcsname\hspace{1mm}}
\makeatother


%%%% Header
\pagestyle{fancy}
\fancyhf{}
\fancyhead[L]{\fontsize{9}{9} \itshape
  INAC 2017 - International Nuclear Atlantic Conference, Belo Horizonte, Brazil,\\
  October 22-27, 2017
}


%%%% Maketitle
\title{Professional cluster management by a small scientific team: challenges, solutions
and perpectives}
%\author{Vitor V. Araújo Silva,$^{*}$ and André A. Campagnole dos Santos and Renan O. da Cunha}
\author{Vitor V. Araújo Silva and André A. Campagnole dos Santos and Renan O. da Cunha}

\institute{
  %$^{*}$Centro de Desenvolvimento da Tecnologia Nuclear - CDTN
  Centro de Desenvolvimento da Tecnologia Nuclear - CDTN
}

\email{\{vitors, aacs, roc\}@cdtn.br}

% Optional disclaimer: remove this command to hide
\disclaimer{Notice: this manuscript is a work of fiction. Any resemblance to
actual articles, living or dead, is purely coincidental.}


%%%% Abstract
\begin{document}
\vspace*{-42pt}
\begin{strip}
\centering{\parbox{153mm}{{\bf Abstract} \itshape - 
The specification, configuration and management of a professional computer cluster are specialized
tasks usually hold by well trained teams, often full-time hired computer scientists. However, in
many situations and for widely different reasons, these very specific technical tasks must
be carried on by no other than the user itself. This is the situation at Centro de Desenvolvimento
da Tecnologia Nuclear - and in many nuclear research and educational centres in developing countries -
where the scientists are the users of the cluster but also the technical
team responsible to keep the system running. This paper presents the process of planning
and installing the whole operational system and scientific software of a professional cluster
aimed to be used in the nuclear engineering field from the point of view of its users.
The drawbacks of lack of expertise and technical skills to
manage such type of technology are opposed to the advantages of freedom to chose the solutions
which best fit to the problems to be solved. The details of selected methods or technologies
chosen for addressing a specific matter are presented together with other possible options, 
offering a broader view of the whole process of cluster's configuration. Specificities
of dealing with closed, restricted and open software, common in the nuclear engineering field,
are also put in perspective. The ideas and solutions presented in this paper can be a
valuable reference to other research teams found in a similar situation:
being scientists and its own technical staff at the same time.


}\par}
\vspace*{14pt}
\end{strip}


%%%%%%%%%%%%%%%%%%%%%%%%%%%%%%%%%%%%%%%%%%%%%%%%%%%%%%%%%%%%%%%%%%%%%%%%%%%%%%%%
\section{Introduction}

Nowdays computers are indispensable tools for scientists of any field.
Some areas of research heavily rely on computational power in order to solve
complex problems by use of demanding algorithms, methods and heuristics.
Current methods used for nuclear reactor calculations, both thermal-hydraulics
and neutronics can be more accurate or even only applicable by using
many cores or computers alltogheter.

A cluster is a...

\subsection{Context}
(Qual será o uso do cluster? Por quem? Quais os perfis dos usuários e de possíveis usuários?
Qual o impacto desse cluster para o CDTN? Em que isso pode nos fortalecer em Computação Científica?)
Um bom artigo num nível acima do nosso \textit{cluster} fala sobre os novos desafios de \textit{hexascale computing}
é o de Dongarra \cite{Dongarra2017}.

Teste de todas as citações disponíveis no bibtex: Wu\cite{Wu2016}, Nagaya\cite{Nagaya2015}.

%%%%%%%%%%%%%%%%%%%%%%%%%%%%%%%%%%%%%%%%%%%%%%%%%%%%%%%%%%%%%%%%%%%%%%%%%%%%%%%%
\section{Objective}

The objective of this paper is to present the challenges, problems and solutions proposed
to setup a professional computational
cluster to carry heavy numerical calculations in a context of a small research team.

(Apresentar as decisões técnicas envolvidas na instalação de um cluster
profissional por pesquisadores/técnicos.)

%%%%%%%%%%%%%%%%%%%%%%%%%%%%%%%%%%%%%%%%%%%%%%%%%%%%%%%%%%%%%%%%%%%%%%%%%%%%%%%%
\section{Methodology}

  A figura~\ref{fig:esquema-cluster} apresenta o sistema de forma simplificada.
  
\begin{itemize}

\item (Indentificação do sistema $\rightarrow$ Características: memória, CPUs, GPUs, cores, rede, topologia, sistema de arquivos, inserção na rede institucional)

\item (Possíveis soluções/soluções encontradas em relação a cada característica apresentada. Por exemplo: Lustre \cite{Lustre} ou Gluster\cite{Gluster} para o sistema de arquivos?)

\item (Para cada solução escolhida: Por quê?)
  
\end{itemize}

The system is formed by a set of eight identical blade servers each with 20 cores and equiped with
NVIDIA GPU's M200 CUDA\cite{CUDA} capable. The GPU's on servers, despite capable of visualization,
are expected to be used as secondary calculation units. Also part of the sistem, the master server is aimed to calculations
visualiation and simulations manager. The master computer has
four network devices and a slightly more powerful GPU to visualization and display.

The original OS sold with the cluster is an OEM (Original Equipament Manufacturer) Windows 7 \cite{windows7}. This
OS is not supported by the whole set of software expected to be used by the research group users. The chosen
system to replace the original Windows 7 is CentOS \cite{centos}, which offers a plataform to natively run all
software needed by the users. However, there are more than one way of setup a cluster system with respect
to filesystem, software execution and user access.

The next paragraphs, describe different option of setup for a Linux cluster with respect to software execution.

\begin{itemize}
\item Direct OS installation
\item Use of dockers
\item SO virtualization
\end{itemize}

%%%%%%%%%%%%%%%%%%%%%%%%%%%%%%%%%%%%%%%%%%%%%%%%%%%%%%%%%%%%%%%%%%%%%%%%%%%%%%%%
\section{Results and Analysis}

\begin{itemize}

\item (Situação atual do sistema - no momento de finalizer o paper)
\item (Dificuldades identificadas: Resolvidas (como) e Pendentes.
  
\end{itemize}

%%%%%%%%%%%%%%%%%%%%%%%%%%%%%%%%%%%%%%%%%%%%%%%%%%%%%%%%%%%%%%%%%%%%%%%%%%%%%%%%
\subsection{Subsection Goes Here}
The user must manually capitalize initial letters of a subsection heading.

Figure~\ref{fig:esquema-cluster}
Note how Fig.~\ref{fig:esquema-cluster} uses dashed lines \verb|--| for the exact
solution, solid lines \verb|-| for the new method's solutions, and dotted lines
\verb|:| for existing inaccurate methods.
\begin{figure}[ht] % replace 't' with 'b' to force it to be on the bottom
  \centering
  \includegraphics[scale=0.15]{images/esquema_cluster_edited_bw.png}
  \caption{Captions are flush with the left.}
  \label{fig:esquema-cluster}
\end{figure}

%%%%%%%%%%%%%%%%%%%%%%%%%%%%%%%%%%%%%%%%
%% \begin{table*}[htb]
%%   \centering
%% \begin{tabular}{llllllllll}\toprule
%%       & $\phi_T(0)$      & $\phi_T(10)$      & $\phi_T(20)$      &
%%       $\phi_D(0)$      & $\phi_D(10)$      & $\phi_D(20)$      & $\rho$      &
%%       $\varepsilon$      & $N_\text{it}$
%% \\ \midrule
%% $c=0.999$  & 0.9038 & 20.63 & 31.24 & 0.9087 & 20.63 & 31.23 & 0.2192 & $10^{-7}$ & 15
%% \\
%% $c=0.990$  & 0.3675 & 13.04 & 24.7 & 0.3696 & 13.04 & 24.69 & 0.2184 & $10^{-7}$ & 15
%% \\
%% $c=0.900$  & 0.009909 & 4.776 & 17.64 & 0.009984 & 4.786 & 17.63 & 0.2118 & $10^{-7}$ & 14
%% \\
%% $c=0.500$  & $6.069\times 10^{-5}$ & 2.212 & 15.53 & 6.213$\times 10^{-5}$ & 2.239 & 15.53 & 0.2068 & $10^{-7}$ & 13
%% \\
%% \bottomrule
%% \end{tabular}
%%   \caption{This is an example of a really wide table which might not normally
%%   fit in the document.}
%%   \label{tab:widetable}
%% \end{table*}
%%%%%%%%%%%%%%%%%%%%%%%%%%%%%%%%%%%%%%%%
%% Notice how the table reference uses a Roman numeral
%% for its numbering scheme, whereas the figure reference uses an Arabic numeral.
%% For one-column tables, use the \verb|table| environment; two-column tables use
%% \verb|table*|. The same applies to figures.

%%%%%%%%%%%%%%%%%%%%%%%%%%%%%%%%%%%%%%%%%%%%%%%%%%%%%%%%%%%%%%%%%%%%%%%%%%%%%%%%
\subsection{Another Subsection}

\subsubsection{Third-level Heading}

%%%%%%%%%%%%%%%%%%%%%%%%%%%%%%%%%%%%%%%%%%%%%%%%%%%%%%%%%%%%%%%%%%%%%%%%%%%%%%%%
\section{Conclusions}

(O que está bom, o que está ruim na atual instalação)
(Trabalhos futuros: pra onde ir (operação), possíveis mudanças)

Uma citação \cite{Henderson17}.

%%%%%%%%%%%%%%%%%%%%%%%%%%%%%%%%%%%%%%%%%%%%%%%%%%%%%%%%%%%%%%%%%%%%%%%%%%%%%%%%
\appendix
\section{Appendix}

%%%%%%%%%%%%%%%%%%%%%%%%%%%%%%%%%%%%%%%%%%%%%%%%%%%%%%%%%%%%%%%%%%%%%%%%%%%%%%%%
\section{Acknowledgments}
The authors would like to thank FUJB for financing the cluster acquisition
as part of the project XXXXX.

%%%%%%%%%%%%%%%%%%%%%%%%%%%%%%%%%%%%%%%%%%%%%%%%%%%%%%%%%%%%%%%%%%%%%%%%%%%%%%%%
\bibliographystyle{ans} % Don't forget to run BibTeX !
\bibliography{bibli}

\end{document}
