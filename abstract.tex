The specification, configuration and management of a professional computer cluster are specialized
tasks usually hold by well trained teams, often full-time hired computer scientists. However, in
many situations and for widely different reasons, these very specific technical tasks must
be carried on by no other than the user itself. This is the situation at Centro de Desenvolvimento
da Tecnologia Nuclear - and in many nuclear research and educational centres in developing countries -
where the scientists are the users of the cluster but also the technical
team responsible to keep the system running. This paper presents the process of planning
and installing the whole operational system and scientific software of a professional cluster
aimed to be used in the nuclear engineering field from the point of view of its users.
The drawbacks of lack of expertise and technical skills to
manage such type of technology are opposed to the advantages of freedom to chose the solutions
which best fit to the problems to be solved. The details of selected methods or technologies
chosen for addressing a specific matter are presented together with other possible options, 
offering a broader view of the whole process of cluster's configuration. Specificities
of dealing with closed, restricted and open software, common in the nuclear engineering field,
are also put in perspective. The ideas and solutions presented in this paper can be a
valuable reference to other research teams found in a similar situation:
being scientists and its own technical staff at the same time.

